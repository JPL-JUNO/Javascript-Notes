\chapter{Vue2基础}
\section{初识Vue}
el用于指定当前Vue实例为哪个容器服务,值通常为CSS选择器字符串。

data中用于存储数据,数据供el所指定的容器去使用,可以写成函数,但是这里先写成一个对象。
\begin{html}
    <script type="text/javascript">
    const x = new Vue({
    el: '#root',
    // el: document.getElementById('root')
    data: {
    name: 'Vue2',
    }
    })
    </script>
\end{html}

\begin{itemize}
    \item 想让Vue工作,就必须创建一个Vue实例,且要传入一个配置对象;
    \item root容器里的代码依然符合html规范,只不过混入了一些特殊的Vue语法;
    \item root容器里的代码被称为Vue模板;
    \item Vue实例和容器是一一对应的;
    \item 真实开发中只有一个Vue实例,并且会配合着组件一起使用;
    \item \{\{xyz\}\}中的xyz要写js表达式,且xyz可以自动读取到data中的所有属性;
    \item 一旦data中的数据发生改变,那么页面中用到该数据的地方也会自动更新;
\end{itemize}
\section{插值语法}
Vue模板语法有2大类:
\begin{enumerate}
    \item 插值语法:
          \begin{itemize}
              \item 功能:用于解析\textbf{标签体}内容;
              \item 写法:\{\{xyz\}\},xyz是js表达式,且可以直接读取到data中的所有属性。
          \end{itemize}
    \item 指令语法:
          \begin{itemize}
              \item 功能:用于解析标签(包括:标签属性、标签体内容、绑定事件等);
              \item 举例:\verb|v-bind:href="xyz"| 或  简写为 \verb|:href="xyz"|,zyx同样要写js表达式,
                    且可以直接读取到data中的所有属性;\zd{(不能有空格)}
              \item 备注:Vue中有很多的指令,且形式都是:\verb|v-xyz|,此处我们只是拿v-bind举个例子;
          \end{itemize}
\end{enumerate}
\begin{html}
    <body>
    <div id="root">
    <h1>Hello, 插值语法</h1>
    <h2>Hello, {{name}}</h2>
    <hr>
    <h1>指令语法</h1>
    <a v-bind:href="url">点击去连接</a>
    <script type="text/javascript">
    new Vue({
    el: '#root',
    data: {
    name: 'Jack',
    url: 'https://www.apple.com.cn'
    }
    })
    </script>
    </div>
    </body>
\end{html}
\section{数据绑定}
Vue2中有2种数据绑定的方式:
\begin{enumerate}
    \item 单向绑定(v-bind):数据只能从data流向页面;
    \item 双向绑定(v-model):数据不仅能从data流向页面,还可以从页面流向data。
\end{enumerate}

\zd{注意点:}

\begin{itemize}
    \item 双向绑定一般都应用在表单类元素上(如:input、select等);
    \item v-model:value 可以简写为 v-model,因为v-model默认收集的就是value值。
\end{itemize}

\begin{html}
    <body>
    <div id="root">
    v-bind: <input type="text" v-bind:value="name"><br>
    v-model: <input type="text" v-model:value="name">
    <br>
    <!-- simplified -->
    v-bind: <input type="text" :value="name"><br>
    v-model: <input type="text" v-model="name">
    </div>
    <script type="text/javascript">
    new Vue({
    el: '#root',
    data: {
    name: 'Vue2',
    }
    })
    </script>
    </body>
\end{html}
\section{el和data的两个中写法}
\begin{enumerate}
    \item el有2种写法
          \begin{itemize}
              \item new Vue时候配置el属性;
              \item 先创建Vue实例,随后再通过\verb|vm.$mount('#root')|指定el的值。
          \end{itemize}
    \item data有2种写法
          \begin{itemize}
              \item 对象式;
              \item 函数式;
          \end{itemize}
    \item 如何选择:目前哪种写法都可以,以后学习到组件时,data必须使用函数式,否则会报错。
    \item 一个重要的原则:\zd{由Vue管理的函数,一定不要写箭头函数,一旦写了箭头函数,this就不再是Vue实例了。}
\end{enumerate}
\begin{html}
    <body>
    <div id="root">
    <h1>Hello, {{name}}</h1>
    </div>
    <script type="text/javascript">
    const v = new Vue({
    // element way1 object
    // el: '#root',
    // data: way1
    // data: {
    //     name: 'Vue2',
    // }
    // data: way2 function
    data: function () {
            // this is the instance object
            console.log('@@@', this)
            return {
                    name: 'Vue2'
                }
        }
    })
    // element way2
    console.log(v);
    v.$mount('#root');
    </script>
    </body>
\end{html}
\section{MVVM模型}
\begin{enumerate}
    \item M:模型(Model) :data中的数据;
    \item V:视图(View) :模板代码;
    \item VM:视图模型(ViewModel):Vue实例;
\end{enumerate}

\verb|data|中所有的属性,最后都出现在了vm(实例对象)身上,vm身上所有的属性以及Vue原型上所有属性,在Vue模板中都可以直接使用。
\begin{html}
    <body>
    <div id="root">
    <h1>Hello, {{name}}</h1>
    <p>{{_c}}</p>
        <p>{{$emit}}</p>
    </div>
    <script type="text/javascript">
    Vue.config.productionTip = false

    const v = new Vue({
            data: function () {
                    return {
                            name: 'Vue2',
                            address: 'uus'
                        }
                }
        }).$mount('#root');
    </script>
    </body>
\end{html}
\section{数据代理}
\subsection{\texttt{object.defineProperty}}
\begin{html}
    <script type="text/javascript">
    let age = 18;
    let person = {
    name: 'John',
    sex: 'male'
    }
    Object.defineProperty(person, 'age', {
            // value: 18,
            // enumerable: true,//can be enumerated or not
            // writable: true,//can be modified or not
            // configurable: true,//can be deleted or not
            get() {
                    console.log('read age');
                    return age
                },
            set(value) {
                    console.log('modify age');
                    age = value;
                }
        })
    // not enumerable
    console.log(Object.keys(person))
    console.log(person)
    </script>
\end{html}
\subsection{数据代理}
数据代理:通过一个对象代理另一个对象中的属性的操作(读/写)。
\begin{html}
    let obj = {
    x: 100
    }
    let obj2 = {
    y: 200
    }
    Object.defineProperty(obj2, 'x', {
            get() {
                    return obj.x
                },
            set(value) {
                    bj.x = value
                }
        })
\end{html}
\section{事件处理}
\subsection{事件的基本使用}
\begin{itemize}
    \item 使用\verb|v-on:xyz|或\verb|@xyz|绑定事件,其中xyz是事件名;
    \item 事件的回调需要配置在methods中,最终会在vm上;
    \item methods中配置的函数,不要使用箭头函数,否则this就是指向vm对象或者组件实例对象;
    \item \verb|@click="demo"|和\verb|@click="demo($click)"|效果一样,但是后者可以传参;
\end{itemize}