\chapter{ECMAscript}
\section{变量}
\begin{table}
\caption{声明变量的特殊情况}
\centering
\begin{tabular}{lll}
\hline
情况&说明&结果\\
\hline
\verb|var age; console.log(age);|&只声明 不赋值&undefined\\
\verb|console.log(age);|&不声明 不赋值&报错\\
\verb|age = 10; console.log(age);|&不声明 只赋值&10\\
\hline
\end{tabular}
\end{table}
\section{数据类型}
JavaScript
\subsection{数据类型分类}
JS把数据类型分为两类:
\begin{itemize}
	\item 简单数据类型(Number、String、Boolean、Undefined、Null)
	\item 复杂数据类型(object)
\end{itemize}
\subsubsection{数字型Number}
\paragraph{数字型进制}
\paragraph{数字型返回}

\paragraph{\texttt{isNaN()}}
这个方法用来判断是否是数字,如果是返回false,反之返回true。(纯数字加引号也会被认定为数字)
\subsubsection{字符串型String}
字符串型可以是引号中的任意文本,可以是单引号或者双引号。

因为HTML标签中里面的属性使用的是双引号,JS这里更推荐使用单引号。
\paragraph{字符串转义字符}
\paragraph{字符串长度}
length
\paragraph{字符串拼接}
字符串+任何类型=字符串
\begin{js}
var age = 19;
console.log('a' + (age + 1) + 'b');
\end{js}

\subsubsection{布尔型Boolean}
\subsubsection{Undefined与Null}





