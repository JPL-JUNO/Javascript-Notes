\chapter{项目笔记}
\begin{js}
    lintOnSave: false,
\end{js}
\section{Element-UI}
\subsection{Form表单}
\paragraph{表单验证}
在防止用户犯错的前提下,尽可能让用户更早地发现并纠正错误。
\paragraph{Form Methods}
\begin{table}[H]
    \centering
    \caption{表单方法}
    \label{formmethods}
    \begin{tabularx}{\textwidth}{lXl}
        \hline
        方法名      & 说明                                                                                     & 参数       \\
        \hline
        validate & 对整个表单进行校验的方法,参数为一个回调函数。该回调函数会在校验结束后被调用,并传入两个参数:是否校验成功和未通过校验的字段。若不传入回调函数,则会返回一个 promise & validate \\
        \hline
    \end{tabularx}

\end{table}

\section{登录和退出功能}
\subsection{登录功能实现}
\subsubsection{路由导航守卫控制访问权限}
如果用户没有登录,但是直接通过URL方位特定页面,需要重新导航到登录页面。
\begin{js}
    // Mount route navigation guard
    // to: the path to be accessed
    // from: Represents which path it jumped from
    // next(): let go, next('/path'): Force jump to path page
    router.beforeEach((to, from, next) => {
    if (to.path === '/login') return next()
    // get token
    const tokenStr = window.sessionStorage.getItem('token')
    if (!tokenStr) return next('/login')
    next()
    })
\end{js}
\subsection{退出功能}
\subsubsection{退出功能原理}
基于token的方式实现退出比较简单,只需要销毁本地的token即可,这样后续的请求就不会携带token,必须重新登录生成一个token之后才可以访问页面。
\begin{js}
    // clear token
    window.sessionStorage.clear()
    // jump to new path
    this.$router.push("/path")
\end{js}
\section{主页布局}
\subsection{通过接口获取菜单数据}

通过axios请求拦截器添加token,保证拥有获取数据的权限。

\section{项目处理步骤}
\begin{enumerate}
    \item 用户列表界面的创建(第三天)
          \begin{itemize}
              \item 添加用户:添加表单、表单内容自定义规则检验、表单重置功能、预验证功能、发起添加用户请求、
          \end{itemize}
    \item \begin{itemize}
              \item
          \end{itemize}
    \item \begin{itemize}
              \item
          \end{itemize}
\end{enumerate}