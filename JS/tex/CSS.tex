\chapter{CSS}
\section{元素显示模式}
元素显示模式就是元素(标签)以什么方式进行显示,比如说div自己占一行,而span却可以一行内写多个。

HTML元素一般分为块元素和行内元素两种类型。

\subsection{块元素}
常见的块元素有h1-h6,p,div,ul,ol,li,其中标签是最典型的块元素。(似乎暗示自带换行的功能? )

块级元素的特点:
\begin{itemize}
\item 自己独占一行;
\item 高度、宽度、外边距预计内边距都可以控制;

\item 宽度默认是容器(父级宽度)的100\%;

\item 是一个容器及盒子,里面可以放行内或者块级元素;
\end{itemize}

注意:
\begin{itemize}
\item 文字类的元素不能使用块级元素;
\item p标签主要用于存放文字,因此里面不能放块级元素;
\item h1-h6都是文字类块级标签,里面也不能放其他块级元素;
\end{itemize}
\subsection{行内元素(内联元素)}
常见的行内元素有a、strong、b、em、i、del、s、ins、u、span等,其中span标签是最典型的行内元素。

行内元素的特点:
\begin{itemize}
\item 相邻行内元素在一行上可以显示多个;
\item 高、宽直接设置是无效的;
\item 默认宽度就是元素本身内容的宽度;
\item 行内元素只能容纳文本或其他行内元素;

\end{itemize}
注意:
\begin{itemize}
\item 链接里面不能再放链接;
\item 特殊情况链接a里面可以放跨级元素,但是给a转换一下块级模式最安全;


\end{itemize}•

