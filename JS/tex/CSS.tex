\chapter{CSS}
\section{元素显示模式}
元素显示模式就是元素(标签)以什么方式进行显示,比如说div自己占一行,而span却可以一行内写多个。

HTML元素一般分为块元素和行内元素两种类型。

\subsection{块元素}
常见的块元素有h1-h6,p,div,ul,ol,li,其中标签是最典型的块元素。(似乎暗示自带换行的功能? )

块级元素的特点:
\begin{itemize}
\item 自己独占一行;
\item 高度、宽度、外边距预计内边距都可以控制;

\item 宽度默认是容器(父级宽度)的100\%;

\item 是一个容器及盒子,里面可以放行内或者块级元素;
\end{itemize}

注意:
\begin{itemize}
\item 文字类的元素不能使用块级元素;
\item p标签主要用于存放文字,因此里面不能放块级元素;
\item h1-h6都是文字类块级标签,里面也不能放其他块级元素;
\end{itemize}
\subsection{行内元素(内联元素)}
常见的行内元素有a、strong、b、em、i、del、s、ins、u、span等,其中span标签是最典型的行内元素。

行内元素的特点:
\begin{itemize}
\item 相邻行内元素在一行上可以显示多个;
\item 高、宽直接设置是无效的;
\item 默认宽度就是元素本身内容的宽度;
\item 行内元素只能容纳文本或其他行内元素;

\end{itemize}
注意:
\begin{itemize}
\item 链接里面不能再放链接;
\item 特殊情况链接a里面可以放跨级元素,但是给a转换一下块级模式最安全;
\end{itemize}

\section{CSS背景}
\subsection{背景图像固定(背景附着)}
background-attachment属性设置背景图像是否是固定或者随着页面
\subsection{背景复合写法}
当使用简写属性时,没有特定的书写顺序,一般习惯按照约定顺序为:

\verb|backgroud: color url repeat attachment position|

\subsection{背景色半透明}
CSS3为我们提供了背景颜色半透明的效果。
\begin{css}
backgroud: rgba(0, 0, 0, .3);
\end{css}


\begin{itemize}
    \item 最后一个参数时alpha透明度,取值范围在$0-1$之间;
    \item 背景半透明是指盒子背景半透明,盒子里面的内容不受影响;
\end{itemize}
\subsection{背景总结}
\begin{table}[H]
    \centering
    \begin{tabular}{lll}
        \hline
        属性&作用&值\\
        \hline
        background-color&背景颜色&预定义的颜色值或十六进制或RGB值\\
        background-image&背景图片&url(path)\\
        background-repeat&是否平铺&repeat/no-repeat/repeat-x/repeat-y\\
        background-position&背景位置&length/position 分别是x和y的坐标\\
        background-attachment&背景附着&scroll(背景滚动) fixed(背景固定)\\
        背景简写&书写更简单&背景颜色 背景图片 背景平铺 背景滚动 背景位置\\
        背景色半透明&背景颜色半透明&background: rgba(red, green, blue, alpha)\\
        \hline 
    \end{tabular}
\end{table}
\section{CSS三大样式}
CSS有三个非常重要的三个特性:层叠性、继承性、优先级。
\subsection{层叠性}
相同选择器设置相同的样式时,此时新的样式就会覆盖另一个冲突的样式,层叠性主要解决样式冲突的问题。

层叠性原则:
\begin{itemize}
    \item 样式冲突,遵顼的原则是就近原则,哪个样式离结构近,就执行哪个样式;
    \item 样式不冲突,不会层叠;
\end{itemize}
\subsection{继承性}
\subsection{优先级}
\autoref{selector weight}显示了选择器的权重
\begin{table}
    \caption{选择器权重}
    \label{selector weight}
    \centering
    \begin{tabular}{ll}
        \hline
        选择器&选择器权重\\
        \hline
        继承或者*&0,0,0,0\\
        元素选择器&0,0,0,1\\
        类选择器、伪类选择器&0,0,1,0\\
        ID选择器&0,1,0,0\\
        行内样式style=""&1,0,0,0\\
        !important&无穷大\\
        \hline
    \end{tabular}
\end{table}
\begin{enumerate}
    \item 继承的权重是0,如果该元素没有直接选中,不管父元素权重多高,子元素得到的权重都是0。
    \item a链接浏览器默认指定列一个样式:蓝色的~有下划线
\end{enumerate}
\subsubsection{权重叠加}
\section{盒子模型}
\subsection{盒子模型的组成}
所谓盒子模型,就是把HTML页面中的布局元素看作是一个矩形的盒子,也就是一个盛装内容的容器。

CSS盒子模型本质是一个盒子,封装周围的HTML元素,它包括:边框(border),外边距(margin),内边距(padding)和实际内容(content)。

\subsubsection{边框(border)}
\paragraph{表格的细线边框}
border-collapse属性控制连六七绘制表格边框的方式,它控制相邻单元格的边框。

语法格式为:
\begin{css}
border-collapse: collapse;
\end{css}
\paragraph{边框会影响盒子的实际大小}
\subsubsection{内边距(padding)}
padding属性用于设置内边距,即边框与内容之间的距离。

如果盒子本身没有指定width/height属性,则此时padding不会撑开盒子大小。
\subsubsection{外边距(margin)}
margin属性用于设置外边距,即控制盒子与盒子之间的距离。
\begin{table}[H]
    \centering
    \begin{tabular}{ll}
        \hline
        属性&作用\\
        \hline
        margin-left&左外边距\\
        margin-right&右外边距\\
        margin-top&上外边距\\
        margin-bottom&下外边距\\
        \hline
    \end{tabular}
\end{table}
\subsubsection{清除外边距}
网页元素很多都带有默认的内外边距,而且不同浏览器默认的值也不一样,因此我们在布局前,首先要清除网页元素的内外边距。

\begin{css}
* {
    padding: 0;
    margin: 0;
}
\end{css}

\textbf{注意:}行内元素尽量只设置左右的内外边距,不要设置上下内外边距。但是转换为块级或者块元素就可以了。