\chapter{CSS}
\section{CSS颜色\label{CCS color}}
\section{元素显示模式}
元素显示模式就是元素(标签)以什么方式进行显示,比如说div自己占一行,而span却可以一行内写多个。

HTML元素一般分为块元素和行内元素两种类型。

\subsection{块元素}
常见的块元素有h1-h6,p,div,ul,ol,li,其中标签是最典型的块元素。(似乎暗示自带换行的功能? )

块级元素的特点:
\begin{itemize}
    \item 自己独占一行;
    \item 高度、宽度、外边距预计内边距都可以控制;

    \item 宽度默认是容器(父级宽度)的100\%;

    \item 是一个容器及盒子,里面可以放行内或者块级元素;
\end{itemize}

注意:
\begin{itemize}
    \item 文字类的元素不能使用块级元素;
    \item p标签主要用于存放文字,因此里面不能放块级元素;
    \item h1-h6都是文字类块级标签,里面也不能放其他块级元素;
\end{itemize}
\subsection{行内元素(内联元素)}
常见的行内元素有a、strong、b、em、i、del、s、ins、u、span等,其中span标签是最典型的行内元素。

行内元素的特点:
\begin{itemize}
    \item 相邻行内元素在一行上可以显示多个;
    \item 高、宽直接设置是无效的;
    \item 默认宽度就是元素本身内容的宽度;
    \item 行内元素只能容纳文本或其他行内元素;

\end{itemize}
注意:
\begin{itemize}
    \item 链接里面不能再放链接;
    \item 特殊情况链接a里面可以放跨级元素,但是给a转换一下块级模式最安全;
\end{itemize}

\section{CSS背景}
\subsection{背景图像固定(背景附着)}
background-attachment属性设置背景图像是否是固定或者随着页面
\subsection{背景复合写法}
当使用简写属性时,没有特定的书写顺序,一般习惯按照约定顺序为:

\verb|backgroud: color url repeat attachment position|

\subsection{背景色半透明}
CSS3为我们提供了背景颜色半透明的效果。
\begin{css}
    backgroud: rgba(0, 0, 0, .3);
\end{css}


\begin{itemize}
    \item 最后一个参数时alpha透明度,取值范围在$0-1$之间;
    \item 背景半透明是指盒子背景半透明,盒子里面的内容不受影响;
\end{itemize}
\subsection{背景总结}
\begin{table}[H]
    \centering
    \begin{tabular}{lll}
        \hline
        属性                    & 作用      & 值                                         \\
        \hline
        background-color      & 背景颜色    & 预定义的颜色值或十六进制或RGB值                         \\
        background-image      & 背景图片    & url(path)                                 \\
        background-repeat     & 是否平铺    & repeat/no-repeat/repeat-x/repeat-y        \\
        background-position   & 背景位置    & length/position 分别是x和y的坐标                 \\
        background-attachment & 背景附着    & scroll(背景滚动) fixed(背景固定)                  \\
        背景简写                  & 书写更简单   & 背景颜色 背景图片 背景平铺 背景滚动 背景位置                  \\
        背景色半透明                & 背景颜色半透明 & background: rgba(red, green, blue, alpha) \\
        \hline
    \end{tabular}
\end{table}
\section{CSS三大样式}
CSS有三个非常重要的三个特性:层叠性、继承性、优先级。
\subsection{层叠性}
相同选择器设置相同的样式时,此时新的样式就会覆盖另一个冲突的样式,层叠性主要解决样式冲突的问题。

层叠性原则:
\begin{itemize}
    \item 样式冲突,遵顼的原则是就近原则,哪个样式离结构近,就执行哪个样式;
    \item 样式不冲突,不会层叠;
\end{itemize}
\subsection{继承性}
\subsection{优先级}
\autoref{selector weight}显示了选择器的权重
\begin{table}
    \caption{选择器权重}
    \label{selector weight}
    \centering
    \begin{tabular}{ll}
        \hline
        选择器          & 选择器权重   \\
        \hline
        继承或者*        & 0,0,0,0 \\
        元素选择器        & 0,0,0,1 \\
        类选择器、伪类选择器   & 0,0,1,0 \\
        ID选择器        & 0,1,0,0 \\
        行内样式style="" & 1,0,0,0 \\
        !important   & 无穷大     \\
        \hline
    \end{tabular}
\end{table}
\begin{enumerate}
    \item 继承的权重是0,如果该元素没有直接选中,不管父元素权重多高,子元素得到的权重都是0。
    \item a链接浏览器默认指定列一个样式:蓝色的~有下划线
\end{enumerate}
\subsubsection{权重叠加}
\section{盒子模型}
\subsection{盒子模型的组成}
所谓盒子模型,就是把HTML页面中的布局元素看作是一个矩形的盒子,也就是一个盛装内容的容器。

CSS盒子模型本质是一个盒子,封装周围的HTML元素,它包括:边框(border),外边距(margin),内边距(padding)和实际内容(content)。

\subsubsection{边框(border)}
\paragraph{表格的细线边框}
border-collapse属性控制连六七绘制表格边框的方式,它控制相邻单元格的边框。

语法格式为:
\begin{css}
    border-collapse: collapse;
\end{css}
\paragraph{边框会影响盒子的实际大小}
\subsubsection{内边距(padding)}
padding属性用于设置内边距,即边框与内容之间的距离。

如果盒子本身没有指定width/height属性,则此时padding不会撑开盒子大小。
\subsubsection{外边距(margin)}
margin属性用于设置外边距,即控制盒子与盒子之间的距离。
\begin{table}[H]
    \centering
    \begin{tabular}{ll}
        \hline
        属性            & 作用   \\
        \hline
        margin-left   & 左外边距 \\
        margin-right  & 右外边距 \\
        margin-top    & 上外边距 \\
        margin-bottom & 下外边距 \\
        \hline
    \end{tabular}
\end{table}
\subsubsection{清除外边距}
网页元素很多都带有默认的内外边距,而且不同浏览器默认的值也不一样,因此我们在布局前,首先要清除网页元素的内外边距。

\begin{css}
    * {
        padding: 0;
        margin: 0;
    }
\end{css}

\textbf{注意:}行内元素尽量只设置左右的内外边距,不要设置上下内外边距。但是转换为块级或者块元素就可以了。
\subsection{圆角边框}
border-radius属性可以设置元素的外边框圆角。

语法格式为:
\begin{css}
    border-radius: length|pct;
\end{css}
\begin{itemize}
    \item 参数值可以为数值或者百分比的形式;
    \item 如果是正方形,想要一个圆,课数值修改为高度或宽度的一般即可,或者直接写50\%;
    \item 如果是一个矩形,设置为高度的一般即可;
    \item 该属性是一个简写属性,可以写四个值,分别表示左上角、右上角、右下角、左下角;
    \item 分开写:border-top-left-radius、border-top-right-radius、border-bottom-right-radius、border-bottom-left-radius;
\end{itemize}
\subsection{盒子阴影(重点)}
box-shadow属性可以为盒子添加阴影。

语法格式为:
\begin{css}
    box-shadow: h-shadow v-shadow blur spread color inset;
\end{css}
\begin{table}[H]
    \centering
    \begin{tabular}{ll}
        \hline
        属性       & 描述                             \\
        \hline
        h-shadow & 必需,水平阴影的位置,允许负值                \\
        v-shadow & 必需,垂直阴影的位置,允许负值                \\
        blur     & 可选,模糊的清晰度                      \\
        spread   & 可选,阴影的尺寸                       \\
        color    & 可选,阴影的颜色,参见\nameref{CCS color} \\
        inset    & 可选,将外部阴影(默认)改为内部阴影             \\
        \hline
    \end{tabular}
\end{table}
\begin{itemize}
    \item 默认为外部阴影(outset),这个值不写,如果写了将导致阴影无效;
    \item 盒子阴影不占用控件,不会印象其他盒子排列;
\end{itemize}
\section{文字阴影(应用不多)}
text-shadow属性将阴影应用于文本。
语法格式为:
\begin{css}
    box-shadow: h-shadow v-shadow blur color;
\end{css}
\begin{table}[H]
    \centering
    \begin{tabular}{ll}
        \hline
        属性       & 描述                             \\
        \hline
        h-shadow & 必需,水平阴影的位置,允许负值                \\
        v-shadow & 必需,垂直阴影的位置,允许负值                \\
        blur     & 可选,模糊的清晰度                      \\
        color    & 可选,阴影的颜色,参见\nameref{CCS color} \\
        \hline
    \end{tabular}
\end{table}
\section{CSS浮动}
\subsection{标准流(普通流/文档流)}
标准流就是标签按照规定好默认的方式排序。
\subsection{浮动}
网页布局第一准则:\zd{多个块级元素纵向排列找标准流,多个块级元素横向排列找浮动。}

float属性用于创建浮动框,将其移动到一边,直到左边缘或右边缘触及包含块或者另一个浮动框的边缘。

语法格式为:
\begin{css}
    selector {
    float: value;
    }
\end{css}
\begin{table}[H]
    \centering
    \begin{tabular}{ll}
        \hline
        属性值   & 描述         \\
        \hline
        none  & 元素不浮动(默认值) \\
        left  & 元素向左浮动     \\
        right & 元素向右浮动     \\
        \hline
    \end{tabular}
\end{table}
\subsection{浮动特性}
\begin{enumerate}
    \item 浮动元素会脱离标准流(拖标),浮动的盒子不再保留原先的位置;
    \item 浮动元素会一行内显示并且元素顶部对齐;
    \item 浮动元素会具有行内块元素的特性
          \begin{itemize}
              \item 如果行内元素有了浮动,则不需要转换为行内块元素就可以直接指定高度和宽度;
              \item 如果块级盒子没有设置宽度,默认宽度和父级宽度一样宽,但是添加浮动后,它的大小根据内容来决定;
              \item 浮动的盒子中间是没有缝隙的,是紧挨在一起的;
              \item 行内元素同理;
          \end{itemize}
\end{enumerate}

为了约束浮动元素的位置,网页布局一般采取的策略是:\zd{先用标准流的父元素排列上下位置,之后内部子元素采取浮动排列左右位置,符合网页布局第一准则。}
\section{常见网页布局}
\section{清除浮动}
清除浮动的本质就是清除浮动元素造成的影响。如果父盒子本身有高度,则不需要清除浮动。清除浮动之后,父级就会根据浮动的子盒子自动检测高度,父级指定高度之后,就不会影响下面的标准流了。

语法格式为:
\begin{css}
    selector{
    clear: value;
    }
\end{css}
\begin{table}[H]
    \centering
    \begin{tabular}{ll}
        \hline
        属性值   & 描述            \\
        \hline
        left  & 元素向左浮动        \\
        right & 元素向右浮动        \\
        both  & 同时清除左右两侧浮动的影响 \\
        \hline
    \end{tabular}
\end{table}
清除浮动的策略是:闭合浮动。
\subsection{清除浮动的方法}
\begin{enumerate}
    \item 额外标签法也称为隔墙法,是W3C推荐的做法;
    \item 父级添加overflow属性;
    \item 父级添加after伪元素;
    \item 父级添加双伪元素;
\end{enumerate}
\paragraph{父级添加overflow属性} 可以给父级添加overflow属性,将其属性值设置为hidden、auto和scroll。代码简洁,但是无法显示溢出内容。

\paragraph{:after伪元素法}

\paragraph{添加双伪元素}
\section{CSS属性书写顺序(重点)}
\begin{enumerate}
    \item 布局定位属性:display/position/float/clear/visibility/overflow;
    \item 自身属性:width/height/margin/padding/border/background;
    \item 文本属性:color/font/text-decoration/text-align/vertical-align/white-space/break-word;
    \item 其他属性(CSS3): content/cursor/border-radius/box-shaow/text-shadow/background: linear-gradient;
\end{enumerate}
\section{页面布局整体思路}
为了提高网页制作的效率, 布局时通常有以下的整体思路:
\begin{enumerate}
    \item 必须确定页面的版心(可视区);
    \item 分析页面中的行模块以及每个行模块中的列模块;
    \item 一行中的列模块经常浮动布局,先确定每个列的大小,之后确定列的位置;
    \item 制作HTML结构,遵循先有结构,后有样式的原则,结构最重要;
\end{enumerate}

\chapter{CSS定位}
浮动可以让多个块级盒子在一行内没有缝隙排列显示,经常用于横向排列盒子。定位则是可以让盒子自由的在某个盒子内移动位置或者固定屏幕中某个位置,并且可以压住其他盒子。
\section{定位组成}
定位是将盒子定在某个位置,所以定位也是摆放盒子,按照定位的方式移动盒子。

定位由定位模式和边偏移两部分组成:
\begin{itemize}
    \item 定位模式用于指定一个元素在文档中的定位方式;
    \item 边偏移决定改元素的最终位置;
\end{itemize}
\subsection{定位模式}
定位模式决定元素的定位方式,通过CSS的\verb|position|属性来设置,可取以下四个值:
\begin{table}[H]
    \centering
    \caption{定位模式属性值}
    \label{positioning mode}
    \begin{tabular}{ll}
        \hline
        属性值             & 描述   \\
        \hline
        \verb|static|   & 静态定位 \\
        \verb|relative| & 相对定位 \\
        \verb|absolute| & 绝对定位 \\
        \verb|fixed|    & 固定定位 \\
        \hline
    \end{tabular}
\end{table}
\subsubsection{静态定位static}
静态定位是元素的默认定位方式,无定位的意思。

语法格式为:
\begin{css}
    selector {
    position: static;
    }
\end{css}

静态定位按照标准流特性拜访位置,它没有边偏移,在布局中很少用到。

\subsubsection{相对定位relative}
相对定位是元素在移动位置的时候,是相对于它原来的位置来参考(自恋型)。

语法格式为:
\begin{css}
    selector {
    position: relative;
    }
\end{css}

特点:
\begin{itemize}
    \item 位置移动参考自己原来所在的位置;
    \item 原来在标准流中的位置继续占有,后面的盒子仍然以标准流的方式对待它(不脱标,继续保留原来位置);
\end{itemize}
\subsubsection{绝对定位absolute}
绝对定位是元素在移动位置的时候,相对于它祖先元素来说的。

语法格式为:
\begin{css}
    selector {
    position: absolute;
    }
\end{css}
特点:
\begin{itemize}
    \item 如果没有祖先元素或者祖先元素没有定位,则以浏览器为准定位(Document文档);
    \item 如果祖先元素有定位,则以最近一级的有定位祖先元素为参考点移动位置;
    \item 绝对定位不占有原先的位置(脱标);
\end{itemize}

\subsubsection{子绝父相口诀}

子绝父相的意思是子级是绝对定位的话,父级要用相对定位。

子级绝对定位,不会占用位置,可以放到父盒子里面的任何一个地方,不会影响其他的兄弟盒子。父盒子需要加定位限制子盒子在父盒子内显示。父盒子布局时,需要占用位置,因此父盒子只能是相对定位。

\subsubsection{固定定位fixed}
固定定位是元素固定于浏览器可视区的位置,主要使用场景是可以在浏览器页面滚动时元素的位置不会改变。
语法格式为:
\begin{css}
    selector {
    position: fixed;
    }
\end{css}

固定定位特点:
\begin{itemize}
    \item 以浏览器的可视窗口为参照点移动元素;
    \item 与父元素没有任何关系;
    \item 不随滚动条滚动;
    \item 固定定位不再占有原先的位置;
\end{itemize}

固定定位是脱标的,其实固定定位也可以看作是一种特殊的绝对定位。

如果想要在版心右侧使用固定定位,首先让固定定位的盒子\verb|left: 50%|,移动到浏览器可视区域的一般位置,再让固定定位的盒子\verb|margin-left: 50% of core|。

\subsection{粘性定位sticky}
粘性定位可以被认为是相对定位和固定定位的混合。

粘性定位语法格式为:
\begin{css}
    selector {
    position: sticky;
    }
\end{css}
粘性定位的特点:
\begin{itemize}
    \item 以浏览器的可视窗口为参考点移动元素(固定定位特点);
    \item 粘性定位占有原先的位置(相对定位特点);
    \item 必须添加\nameref{edgeOffset}中的一个;
\end{itemize}

\subsection{边偏移\label{edgeOffset}}
边偏移决定定位的盒子移动的最终位置,主要由top、bottom、left、right4个属性:
\begin{table}[H]
    \centering
    \caption{边偏移属性}
    \label{edgeOffsetTable}
    \begin{tabular}{lll}
        \hline
        属性值    & 示例           & 描述                     \\
        \hline
        top    & top: 80px    & 顶端偏移量,定义元素相对于父元素上边线的距离 \\
        bottom & bottom: 80px & 底部偏移量,定义元素相对于父元素上边线的距离 \\
        left   & left: 80px   & 左侧偏移量,定义元素相对于父元素左边线的距离 \\
        right  & right: 80px  & 右侧偏移量,定义元素相对于父元素右边线的距离 \\
        \hline
    \end{tabular}
\end{table}
\subsection{定位的总结}
\begin{table}[H]
    \centering
    \caption{定位总结}
    \begin{tabular}{llll}
        \hline
        定位模式         & 是否脱标    & 移动位置      & 是否常用  \\
        \hline
        静态定位static   & 否       & 不能使用边偏移   & 很少    \\
        相对定位relative & 否(占有位置) & 相对于自身位置移动 & 常用    \\
        绝对定位absolute & 是(不占位置) & 带有定位的父级   & 常用    \\
        固定定位fixed    & 是(不占位置) & 浏览器可视区    & 常用    \\
        粘性定位sticky   & 否(占有位置) & 浏览器可视区    & 当前阶段少 \\
        \hline
    \end{tabular}
\end{table}
\subsection{定位叠放次序z-index}
在使用定位布局时,可能会出现盒子重叠的情况,此时可以使用z-index来控制盒子的前后次序(z轴)。

语法格式为:
\begin{css}
    selector {
    z-index: 1;
    }
\end{css}

数值可以是正整数、负整数或者0,默认是auto,数值越大、盒子越靠上。如果属性值相同,则按照章节书写顺序,后来居上,数字后面不能加单位,并且只有有定位属性的盒子才有z-index属性。
\subsection{定位的拓展}
\subsubsection{绝对定位盒子的水平垂直居中方法}
如果想要在版心右侧使用固定定位,首先让固定定位的盒子\verb|left: 50%|,移动到浏览器可视区域的一般位置,再让固定定位的盒子\verb|margin-left: -50% of width|。

\subsubsection{定位特殊特性}
\begin{itemize}
    \item 行内元素添加绝对或固定定位,可以直接设置高度和宽度;
    \item 块级元素添加绝对或者固定定位,如果不该宽度或者高度,默认大小是内容的大小;
\end{itemize}
\subsubsection{绝对定位/固定定位会完全压住盒子}
浮动元素是不同的,尽会压住它下面标准流的盒子,但是不会压住下面标准流盒子里面的文字以及图片,但是绝对定位(固定定位)会压住下面标准流所有的内容。

浮动之所以不会压住文字,是因为浮动产生的最初目的是为了左文字环绕效果的,文字会围绕浮动元素。



