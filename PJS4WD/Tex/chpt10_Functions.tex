\chapter{函数}
每个函数都是Function类型的实例,而 Function 也有属性和方法,跟其他引用类型一样。因为函数是对象,所以函数名就是指向函数对象的指针,而且不一定与函数本身紧密绑定。

函数通常以函数声明的方式定义:
\begin{js}
function sum(num1, num2) {
    return num1 + num2;
}
\end{js}
注意函数定义最后没有加分号。

另一种定义函数的语法是函数表达式:
\begin{js}
let sum = function(num1, num2) {
    return num1 + num2;
};
\end{js}

代码定义了一个变量 sum 并将其初始化为一个函数。注意 function 关键字后面没有名称,因为不需要。这个函数可以通过变量 sum 来引用。 注意这里的函数末尾是有分号的,与任何变量初始化语句一样。

还有一种定义函数的方式与函数表达式很像,叫作“箭头函数”(arrow function):
\begin{js}
let sum = (num1, num2) => {
    return num1 + num2;
};
\end{js}

最后一种定义函数的方式是使用 Function 构造函数。这个构造函数接收任意多个字符串参数,最后一个参数始终会被当成函数体,而之前的参数都是新函数的参数:
\begin{js}
let sum = new Function("num1", "num2", "return num1 + num2")
\end{js}

我们不推荐使用这种语法来定义函数,因为这段代码会被解释两次:第一次是将它当作常规ECMAScript 代码,第二次是解释传给构造函数的字符串。这显然会影响性能。不过,把函数想象为对象,把函数名想象为指针是很重要的。

\cautions{
    这几种实例化函数对象的方式之间存在微妙但重要的差别,本章后面会讨论。无论如何,通过其中任何一种方式都可以创建函数。
}

\section{箭头函数}
ECMAScript 6 新增了使用胖箭头($=\textgreater$)语法定义函数表达式的能力。很大程度上,箭头函数实例化的函数对象与正式的函数表达式创建的函数对象行为是相同的。任何可以使用函数表达式的地方,都可以使用箭头函数:

箭头函数简洁的语法非常适合嵌入函数的场景:
\begin{js}
let ints = [1, 2, 3];

console.log(
  ints.map(function (i) {
    return i + 1;
  })
);
console.log(
  ints.map((i) => {
    return i + 1;
  })
);
\end{js}
如果只有一个参数,那也可以不用括号。只有没有参数,或者多个参数的情况下,才需要使用括号:
\begin{js}
let double = (x) => {
  return 2 * x;
};
let tripe = (x) => {
  return 3 * x;
};
// Zero parameters require an empty pair of parentheses
let getRandom = () => {
  return Math.random();
};
// Multiple parameters require parentheses
let sum = (a, b) => {
  return a + b;
};
// Invalid syntax:
let multiply = a, b => { 
    return a * b;
};
\end{js}
箭头函数也可以不用大括号,但这样会改变函数的行为。使用大括号就说明包含“函数体”,可以在一个函数中包含多条语句,跟常规的函数一样。如果不使用大括号,那么箭头后面就只能有一行代码,比如一个赋值操作,或者一个表达式。而且,省略大括号会隐式返回这行代码的值:
\begin{js}
let double = (x) => {
  return 2 * x;
};
let tripe = (x) => 3 * x;

let value = {};
let setName = (x) => (x.name = "Matt");
setName(value);
console.log(value.name);

// Invalid syntax:
let multiply = (a, b) => return a * b;
\end{js}

箭头函数虽然语法简洁,但也有很多场合不适用。箭头函数不能使用 arguments、super 和 new.target,也不能用作构造函数。此外,箭头函数也没有 prototype 属性。

\section{函数名}
因为函数名就是指向函数的指针,所以它们跟其他包含对象指针的变量具有相同的行为。这意味着一个函数可以有多个名称:
\begin{js}
function sum(num1, num2) {
  return num1 + num2;
}

console.log(sum(10, 10));

let anotherSum = sum;
console.log(anotherSum(10, 10));

sum = null;
console.log(anotherSum(10, 10));
\end{js}

声明了一个变量 anotherSum,并将它的值设置为等于 sum。注意,使用不带括号的函数名会访问函数指针,而不会执行函数。此时, anotherSum 和 sum 都指向同一个函数。调用 anotherSum()也可以返回结果。把 sum 设置为 null 之后,就切断了它与函数之间的关联。而 anotherSum()还是可以照常调用,没有问题。

ECMAScript 6 的所有函数对象都会暴露一个只读的 name 属性,其中包含关于函数的信息。多数情况下,这个属性中保存的就是一个函数标识符,或者说是一个字符串化的变量名。即使函数没有名称,也会如实显示成空字符串。如果它是使用 Function 构造函数创建的,则会标识成“anonymous”:
\begin{js}
function foo() {}
let bar = function () {};
let baz = () => {};

console.log(foo.name); // foo
console.log(bar.name); // bar
console.log(baz.name); // baz
console.log((() => {}).name); // (empty string)
console.log(new Function().name); // anonymous
\end{js}

如果函数是一个获取函数、设置函数,或者使用 bind()实例化,那么标识符前面会加上一个前缀:
\begin{js}
function foo() {}

console.log(foo.bind(null).name); // bound foo
let dog = {
  years: 1,
  get age() {
    return this.years;
  },
  set age(newAge) {
    this.years = newAge;
  },
};

let propertyDescriptor = Object.getOwnPropertyDescriptor(dog, "age");
console.log(propertyDescriptor.get.name); // get age
console.log(propertyDescriptor.set.name); // set age
\end{js}

\section{理解参数}
ECMAScript 函数的参数跟大多数其他语言不同。ECMAScript 函数既不关心传入的参数个数,也不关心这些参数的数据类型。定义函数时要接收两个参数,并不意味着调用时就传两个参数。你可以传一个、三个,甚至一个也不传,解释器都不会报错。之所以会这样,主要是因为 ECMAScript 函数的参数在内部表现为一个数组。函数被调用时总会接收一个数组,但函数并不关心这个数组中包含什么。如果数组中什么也没有,那没问题;如果数组的元素超出了要求,那也没问题。

事实上,在使用 function 关键字定义(非箭头)函数时,可以在函数内部访问 arguments 对象,从中取得传进来的每个参数值。arguments 对象是一个类数组对象(但不是 Array 的实例),因此可以使用中括号语法访问其中的元素(第一个参数是 arguments[0],第二个参数是 arguments[1])。而要确定传进来多少个参数,可以访问 arguments.length 属性。

\begin{js}
function sayHi(name, message) {
  console.log("Hello " + name + "," + message);
}

function sayHi() {
  console.log("Hello" + arguments[0] + ", " + arguments[1]);
}
\end{js}

在重写后的代码中,没有命名参数。name 和 message 参数都不见了,但函数照样可以调用。这就表明,ECMAScript 函数的参数只是为了方便才写出来的,并不是必须写出来的。与其他语言不同,在ECMAScript 中的命名参数不会创建让之后的调用必须匹配的函数签名。这是因为根本不存在验证命名参数的机制。

\begin{js}
function howManyArgs() {
  console.log(arguments.length);
}

howManyArgs("string", 45); // 2
howManyArgs(); // 0
\end{js}
既然如此,那么开发者可以想传多少参数就传多少参数。比如:
\begin{js}
function doAdd() {
  if (arguments.length === 1) {
    console.log(arguments[0] + 10);
  } else if (arguments.length === 2) {
    console.log(arguments[0] + arguments[1]);
  }
}

doAdd(10); // 20
doAdd(30, 20); // 50
\end{js}

还有一个必须理解的重要方面,那就是 arguments 对象可以跟命名参数一起使用:

\begin{js}
function doAdd(num1, num2) {
  if (arguments.length === 1) {
    console.log(num1 + 10);
  } else if (arguments.length === 2) {
    console.log(arguments[0] + num2);
  }
}
\end{js}

arguments 对象的另一个有意思的地方就是,它的值始终会与对应的命名参数同步:
\begin{js}
function doAdd(num1, num2) {
  arguments[1] = 10;
  console.log(num1 + num2);
}

doAdd(60, 50); // 70
\end{js}
因为 arguments 对象的值会自动同步到对应的命名参数,所以修改 arguments[1]也会修改 num2 的值,因此两者的值都是 10。但这并不意味着它们都访问同一个内存地址,它们在内存中还是分开的,只不过会保持同步而已。\zd{另外还要记住一点:如果只传了一个参数,然后把 arguments[1]设置为某个值,那么这个值并不会反映到第二个命名参数。这是因为 arguments 对象的长度是根据传入的参数个数,而非定义函数时给出的命名参数个数确定的。}

严格模式下,arguments 会有一些变化。首先,像前面那样给 arguments[1]赋值不会再影响 num2的值。就算把 arguments[1]设置为 10,num2 的值仍然还是传入的值。其次,在函数中尝试重写 arguments 对象会导致语法错误。(代码也不会执行。

\subsection{箭头函数中的参数}
如果函数是使用箭头语法定义的,那么传给函数的参数将不能使用 arguments 关键字访问,而只能通过定义的命名参数访问。
\begin{js}
function foo() {
  console.log(arguments[0]);
}
foo(5);

let bar = () => {
  console.log(arguments[0]);
};
bar(5); // ReferenceError: arguments is not defined
\end{js}

虽然箭头函数中没有 arguments 对象,但可以在包装函数中把它提供给箭头函数:
\begin{js}
function foo() {
  let bar = () => {
    console.log(arguments[0]); // 7
  };
  bar();
}
foo(7);
\end{js}

\cautions{
    ECMAScript 中的所有参数都按值传递的。不可能按引用传递参数。如果把对象作为参数传递,那么传递的值就是这个对象的引用。
}
\section{没有重载}
ECMAScript 函数不能像传统编程那样重载。在其他语言比如 Java 中,一个函数可以有两个定义,只要签名(接收参数的类型和数量)不同就行。如前所述,ECMAScript 函数没有签名,因为参数是由包含零个或多个值的数组表示的。没有函数签名,自然也就没有重载。\textbf{如果在 ECMAScript 中定义了两个同名函数,则后定义的会覆盖先定义的。}
\begin{js}
function addSomeNumber(num) {
  return num + 100;
}

function addSomeNumber(num) {
  return num + 200;
}

let result = addSomeNumber(500);
console.log(result); // 700
\end{js}

把函数名当成指针也有助于理解为什么 ECMAScript 没有函数重载。在前面的例子中,定义两个同名的函数显然会导致后定义的重写先定义的。而那个例子几乎跟下面这个是一样的:
\begin{js}
let addSomeNumber = function (num) {
  return num + 100;
};

addSomeNumber = function (num) {
  return num + 200;
};

let result = addSomeNumber(500);
console.log(result); // 700
\end{js}

\section{默认参数值}
在 ECMAScript5.1 及以前,实现默认参数的一种常用方式就是检测某个参数是否等于 undefined,如果是则意味着没有传这个参数,那就给它赋一个值:
\begin{js}
function makeKing(name) {
  name = typeof name !== "undefined" ? name : "Henry";
  return `King ${name} VIII`;
}

console.log(makeKing()); // King Henry VIII
console.log(makeKing("Louis")); // King Louis VIII
\end{js}
ECMAScript 6 之后就不用这么麻烦了,因为它支持显式定义默认参数了:
\begin{js}
function makeKing(name = "Henry") {
  return `King ${name} VIII`;
}

console.log(makeKing("Louis")); // King Louis VIII
console.log(makeKing()); // King Henry VIII
\end{js}

给参数传 undefined 相当于没有传值,不过这样可以利用多个独立的默认值:
\begin{js}
function makeKing(name = "Henry", numerals = "VIII") {
  return `King ${name} ${numerals}`;
}

console.log(makeKing()); // King Henry VIII
console.log(makeKing("Louis")); // King Louis VIII
console.log(makeKing(undefined, "VI")); // King Henry VI
console.log(makeKing("VI")); // King VI VIII
\end{js}

\textbf{在使用默认参数时,arguments 对象的值不反映参数的默认值,只反映传给函数的参数}。当然,跟 ES5 严格模式一样,修改命名参数也不会影响 arguments 对象,它始终以调用函数时传入的值为准:
\begin{js}
function makeKing(name = "Henry") {
  name = "Louis";
  return `King ${arguments[0]}`;
}

console.log(makeKing()); // King undefined
console.log(makeKing("Louis")); // King Louis
\end{js}

默认参数值并不限于原始值或对象类型,也可以使用调用函数返回的值:
\begin{js}
let romanNumerals = ["I", "II", "III", "IV", "V", "VI"];
let ordinality = 0;

function getNumerals() {
  // Increment the ordinality after using it to index into the numerals array
  return romanNumerals[ordinality++];
}

function makeKing(name = "Henry", numerals = getNumerals()) {
  return `King ${name} ${numerals}`;
}

console.log(makeKing()); // King Henry I
console.log(makeKing("Louis", "XVI")); // King Louis XVI
console.log(makeKing()); // King Henry II
console.log(makeKing()); // King Henry III
\end{js}

函数的默认参数只有在函数被调用时才会求值,不会在函数定义时求值。而且,计算默认值的函数只有在调用函数但未传相应参数时才会被调用。

箭头函数同样也可以这样使用默认参数,只不过在只有一个参数时,就必须使用括号而不能省略了:
\begin{js}
// The value of this line will implicitly return
let makeKing = (name = "Henry", numerals = "III") => `King ${name} ${numerals}`;

console.log(makeKing()); // King Henry III
\end{js}

\subsection{默认参数作用域与暂时性死区}
因为在求值默认参数时可以定义对象,也可以动态调用函数,所以函数参数肯定是在某个作用域中求值的。给多个参数定义默认值实际上跟使用 let 关键字顺序声明变量一样。因为参数是按顺序初始化的,所以后定义默认值的参数可以引用先定义的参数。
\begin{js}
function makeKing1(name = "Henry", numerals = "VIII") {
  return `King ${name} ${numerals}`; // King Henry VIII
}

console.log(makeKing1());

function makeKing2() {
  let name = "Henry";
  let numerals = "VIII";
  return `King ${name} ${numerals}`;
}

function makeKing3(name = "Henry", numerals = name) {
  return `King ${name} ${numerals}`;
}
console.log(makeKing3()); // King Henry Henry
\end{js}

\verb|makeKing1| 的过程可以依照 \verb|makeKing2| 的过程类似。

参数初始化顺序遵循“暂时性死区”规则,即前面定义的参数不能引用后面定义的参数(前面不能用后面的)。像这样就会抛
出错误:
\begin{js}
function makeKing(name = numerals, numerals = "VIII") {
  return `King ${name} ${numerals}`;
}
console.log(makeKing()); // ReferenceError: Cannot access 'numerals'
\end{js}

参数也存在于自己的作用域中,它们不能引用函数体的作用域:
\begin{js}
function makeKing(name = "Henry", numerals = defaultNumber) {
  let defaultNumber = "VIII";
  return `King ${name} ${numerals}`;
}
console.log(makeKing()); // ReferenceError: defaultNumber is not defined
\end{js}

\section{参数扩展与收集}
ECMAScript 6 新增了扩展操作符,使用它可以非常简洁地操作和组合集合数据。扩展操作符最有用的场景就是函数定义中的参数列表,在这里它可以充分利用这门语言的弱类型及参数长度可变的特点。扩展操作符既可以用于调用函数时传参,也可以用于定义函数参数。
\subsection{扩展参数(解决数组作为参数传入的方式)}
在给函数传参时,有时候可能不需要传一个数组,而是要分别传入数组的元素。

假设有这样一个函数:
\begin{js}
let values = [1, 2, 3, 4];
function getSum() {
    let sum = 0;
    for (let i = 0; i < arguments.length; i++) {
    sum += arguments[i];
    }
    return sum;
}
\end{js}

这个函数希望将所有加数逐个传进来,然后通过迭代 arguments 对象来实现累加。如果不使用扩展操作符,想把定义在这个函数这面的数组拆分,那么就得求助于 apply()方法:
\begin{js}
console.log(getSum.apply(null, values)); // 10
\end{js}
但在 ECMAScript 6 中,可以通过扩展操作符极为简洁地实现这种操作。\zd{对可迭代对象应用扩展操作符,并将其作为一个参数传入,可以将可迭代对象拆分,并将迭代返回的每个值单独传入。}

因为数组的长度已知,所以在使用扩展操作符传参的时候,并不妨碍在其前面或后面再传其他的值,包括使用扩展操作符传其他参数
\begin{js}
console.log(getSum(-1, ...values)); // 9
console.log(getSum(...values, 5)); // 15
console.log(getSum(-1, ...values, 5)); // 14
console.log(getSum(...values, ...[5, 6, 7])); // 28
\end{js}

对函数中的 arguments 对象而言,它并不知道扩展操作符的存在,而是按照调用函数时传入的参数接收每一个值:
\begin{js}
let values = [1, 2, 3, 4];
function countArguments() {
  console.log(arguments.length);
}

console.log(getSum(-1, ...values)); // 5
console.log(getSum(...values, 5)); // 5
console.log(getSum(-1, ...values, 5)); // 6
console.log(getSum(...values, ...[5, 6, 7])); // 7
\end{js}

arguments 对象只是消费扩展操作符的一种方式。在普通函数和箭头函数中,也可以将扩展操作符用于命名参数,当然同时也可以使用默认参数:
\begin{js}
function getProduct(a, b, c = 1) {
  return a * b * c;
}

let getSum = function (a, b, c = 0) {
  return a + b + c;
};

console.log(getProduct(...[1, 2])); // 2
console.log(getProduct(...[1, 2, 3])); // 6
console.log(getProduct(...[1, 2, 3, 4])); // 6

console.log(getSum(...[0, 1])); // 1
console.log(getSum(...[0, 1, 2])); // 3
console.log(getSum(...[0, 1, 2, 3])); // 3
\end{js}
\subsection{收集参数}
在构思函数定义时,可以使用扩展操作符把不同长度的独立参数组合为一个数组。这有点类似arguments 对象的构造机制,只不过收集参数的结果会得到一个 Array 实例。

\begin{js}
function getSum(...values) {
  return values.reduce((x, y) => x + y);
}
console.log(getSum(1, 2, 3));
\end{js}

收集参数的前面如果还有命名参数,则只会收集其余的参数;如果没有则会得到空数组。因为收集参数的结果可变,所以只能把它作为最后一个参数:
\begin{js}
// Error
function getProduct(...values, lastValue) { }

// OK
function ignoreFirst(firstValue, ...values) {
  console.log(values);
}
ignoreFirst(); // []
ignoreFirst(1); // []
ignoreFirst(1, 2); // [2]
ignoreFirst(1, 2, 3); // [2, 3]
\end{js}

箭头函数虽然不支持 arguments 对象,但支持收集参数的定义方式,因此也可以实现与使用arguments 一样的逻辑:
\begin{js}
let getSum = (...values) => {
  return values.reduce((x, y) => x + y, 0);
};

console.log(getSum(1, 2, 3)); // 6
\end{js}

另外,使用收集参数并不影响 arguments 对象,它仍然反映调用时传给函数的参数:
\begin{js}
function getSum(...values) {
  console.log(arguments.length); // 3
  console.log(arguments); // [1, 2, 3]
  console.log(values); // [1, 2, 3]
}

getSum(1, 2, 3);
\end{js}
\section{函数声明与函数表达式}