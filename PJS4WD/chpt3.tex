\chapter{语言基础}
任何语言的核心所描述的都是这门语言在最基本的层面上如何工作,涉及语法、操作符、数据类型
以及内置功能,在此基础之上才可以构建复杂的解决方案。
\section{语法}
ECMAScript 的语法很大程度上借鉴了 C 语言和其他类 C 语言,如 Java 和 Perl。
\subsection{区分大小写}
ECMAScript 中一切都区分大小写。无论是变量、函数名还是操作符,都区分大小写。
\subsection{标识符}
所谓标识符,就是变量、函数、属性或函数参数的名称。标识符可以由一或多个下列字符组成:
\begin{itemize}
    \item 第一个字符必须是一个字母、下划线(\_)或美元符号(\$);
    \item 剩下的其他字符可以是字母、下划线、美元符号或数字。
\end{itemize}
标识符中的字母可以是扩展 ASCII(Extended ASCII)中的字母,也可以是 Unicode 的字母字符。

按照惯例,ECMAScript 标识符使用驼峰大小写形式,即第一个单词的首字母小写,后面每个单词的首字母大写。

\zd{关键字、保留字、true、false 和 null 不能作为标识符。具体内容参考\nameref{keywords}。}

\subsection{注释}
ECMAScript 采用 C 语言风格的注释,包括单行注释和块注释。单行注释以两个斜杠字符开头。块注释以一个斜杠和一个星号(/*)开头,以它们的反向组合(*/)结尾。
\subsection{严格模式}
严格模式是一种不同的 JavaScript 解析和执行模型,ECMAScript 3 的一些不规范写法在这种模式下会被处理,对于不安全的活动将抛出错误。要对整个脚本启用严格模式,在脚本开头加上这一行:
\begin{js}
    "use strict";
\end{js}
\subsection{语句}
\section{关键字和保留字\label{keywords}}